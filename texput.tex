%%% Local Variables:
%%% mode: latex
%%% TeX-master: "<none>"
%%% End:

\title{Symbol encoding}

\author{Vicente González Ruiz}

\maketitle

\section{How it works?}
\begin{itemize}
\item
  We can compress a sequence of symbols if each one is translated by a
  code-word and, in average, the lengths of the code-words are smaller
  than the length of the symbols.
\item
  The encoder and the decoder have a probabilistic model \(M\) which
  provides to a variable-length encoder (\(C\))/decoder(\(C^{-1}\)) the
  probability \(p(s)\) of each symbol \(s\).
\item
  The most probable symbols are represented by the shorter code-words
  and viceversa.
\end{itemize}

\img{600}{graphics/compresion_entropica.png}

\section{Bits, data and information}
\begin{itemize}
\item
  data != information (data is the representation of the information).
\item
  Lossless data compression uses a shorter representation for
  information.
\item
  By definition, a bit of data stores a bit of information, if and only
  if, it represents the occurrence of an equiprobable event (an event
  that can be true or false with the same probability). In this ideal
  situation, the representation is fully efficient (no futher
  compression would be possible).
\item
  By definition, a symbol \(s\) with probability \(p(s)\) stores
  \begin{equation}
    I(s)=-\log_2 p(s) \label{Eq:symbol_information} \tag{Eq:symbol\_information}
  \end{equation}
  bits of information.
\item
  So, ideally, the length of a code-word in bits (of data) should match
  with the number bits of information.
\end{itemize}

\img{600}{graphics/prob_vs_long.png}

\section{Entropy}
\begin{itemize}
\item
  The entropy \(H(S)\) measures the amount of information per symbol
  that a source of information \(S\) produces, in average. By definition

  \begin{equation}
    H(S) = \frac{1}{N}\sum_{s=1}^{N} p(s)\times I(s)
  \end{equation}

  bits-of-information/symbol, where \(N\) is the size of the source
  alphabet (number of different symbols).
\end{itemize}

\section{Compression basics}

\subsection{Encoding of a symbol}
\begin{enumerate}
\def\labelenumi{\arabic{enumi}.}
\tightlist
\item
  While the decoder does not know the symbol:

  \begin{enumerate}
  \def\labelenumii{\arabic{enumii}.}
  \tightlist
  \item
    Assert something about the symbol that allows to the decoder to
    minimize the uncertainty of finding that symbol. This guess should
    have true or false with the same probability.
  \item
    Output a bit of code that says if the last guess is true or false.
  \end{enumerate}
\end{enumerate}

\subsection{Decoding of a symbol}
\begin{enumerate}
\def\labelenumi{\arabic{enumi}.}
\tightlist
\item
  While the symbol is not known without uncertainty:

  \begin{enumerate}
  \def\labelenumii{\arabic{enumii}.}
  \tightlist
  \item
    Make the same guess that the encoder.
  \item
    Input a bit of code that represents the result of the last guess.
  \end{enumerate}
\end{enumerate}

\subsubsection{Tip}
\begin{itemize}
\tightlist
\item
  This codec is 100\% efficient if the guesses are equiprobable.
\end{itemize}
\subsection{Example}
\begin{itemize}
\item
  Let's suppose that we use the Spanish alphabet. Humans know that
  symbols does not form words in any order, so we can formulate the
  following VLC (Variable Length Codec):

  In Spanish there are 28 letters. Therefore, to encode, for example,
  the word \texttt{preciosa}, the first symbol \texttt{p} can be
  represented by its index inside of the Spahish alphabet with a
  code-word of 5 bits. In this try, the encoding is not a very
  efficient, but this we are in first letter \ldots{} For the second one
  \texttt{r} we can see (using a Spanish dictionary) that after a
  \texttt{p}, the following symbols are possible: (1) \texttt{a}, (2)
  \texttt{e}, (3) \texttt{i}, (4) \texttt{l}, (5) \texttt{n}, (6)
  \texttt{o}, (7) \texttt{r}, (8) \texttt{s} and (9) \texttt{u}.
  Therefore, we don't need 5 bits now, 4 are enough.
\end{itemize}

\img{400}{graphics/universal_coding_example.png}

\begin{itemize}
\tightlist
\item
  Notice that the compression ratio has been 40/25:1 (\texttt{preciosa}
  has 8 letters).
\end{itemize}

\bibliography{text-compression}
